\documentclass[11pt]{article}
\usepackage{outline}
\usepackage{pmgraph}
\usepackage{listings}
\usepackage[utf8]{inputenc}
\usepackage[normalem]{ulem}
\usepackage{amsmath}
\usepackage{algorithm}
\usepackage[noend]{algpseudocode}
\usepackage{caption}
\usepackage{mathtools}
\def\BState{\State\hskip-\ALG@thistlm}
\makeatother
\renewcommand{\algorithmicrequire}{\textbf{Input:}}
\renewcommand{\algorithmicensure}{\textbf{Output:}}
\DeclarePairedDelimiter\ceil{\lceil}{\rceil}
\DeclarePairedDelimiter\floor{\lfloor}{\rfloor}
\title{Programming Assignment 1\\ \vspace{10px} \large Heuristic Optimization Techniques}
\author{David Molnar, 1326891\\ Daniel Füvesi, 1326576\\Gruppe 9}
\date{\today}
%--------------------Make usable space all of page
\setlength{\oddsidemargin}{0in}
\setlength{\evensidemargin}{0in}
\setlength{\topmargin}{0in}
\setlength{\headsep}{-.25in}
\setlength{\textwidth}{6.5in}
\setlength{\textheight}{8.5in}
%--------------------Indention
\setlength{\parindent}{0.5cm}

\begin{document}
\lstset{language=Java}
%--------------------Title Page
\maketitle
 
%--------------------Begin Outline
\section{Construction Heuristic}
\subsection{Deterministic Construction Heuristic}

\hspace{0.5cm} Our construction heuristic consists of 2 separate steps, the spine odering and the edge partitioning. For the spine ordering we implemented a simple depth-first search algorithm. To satisfy determinism, the middle node of the initial spine is picked as the root of the new ordering. Neighbours with the lowest indices are prioritised and added to the spine. This new ordering is then used for the edge partitioning, namely the \textit{CFL (Conflicting Edge Distribution)}. Instead of simply iterating through the edges, we first sort them by the number of conflicts they induce in an ascending order and remove the ones with 0 conflicts. The idea is, that we want to move edges that cause the most crossings to another page as soon as possible. We iterate through this set and each edge gets placed on a page where it leads to the least crossings amongst all pages.\\
% With the help of an efficient calculation for the result of a "move", the whole algorithm finishes under a minute.\\
\begin{algorithm}
	\caption{KPMP}\label{euclid}
	\begin{algorithmic}[1]
		\Require KPMPM Problem Instance
		\Ensure KPMPSolution
		\Function{Solve}{ }
		\State $\textit{spine order} \gets \text{\textit{DFS Spine Ordering}(adjacency list,$\floor*{\text{number of vertices} / 2}$)};$
		\State $\textit{edge partitioning} \gets \text{\textit{CFL Edge Partitioning}($\textit{calculated spine order}$, edges)};$
		
		\Return $\text{solution(\textit{spine order, edge partitoning})};$
		\EndFunction
		\newline
		\Function{DFS Spine Ordering}{adjacency list, root vertex index}
		
		\Return $\textit{spine order} \gets \text{DFS(\textit{spine order}, vertex index)};$
		\EndFunction
	
		\algstore{myalg}
	\end{algorithmic}
\end{algorithm}
\begin{algorithm}
	\ContinuedFloat
	\begin{algorithmic}
		
		\algrestore{myalg}
		
		\Function{CFL Edge Partitioning}{spine order, edge list}
		\State $\textit{organizedEdges} \gets \text{\textit{removeEdgesWithZeroCrossings}(edge list)};$
		\State $\textit{organizedEdges} \gets \text{\textit{sortByCrossings}(edge list)};$
		\While{not iterated through all edges in \textit{edges}}

		\State$\textit{edge} \gets \text{next edge in \textit{organizedEdges}} $
		
		\State$\text{move \textit{edge} in \textit{edge list} to page where it leads to the least crossings amongst all pages}$
		\EndWhile
		
		\Return $\textit{edge list}$
		\EndFunction
	\end{algorithmic}
\end{algorithm}
\pagebreak
\newline
A valid solution is encapsuled in an object storing an arraylist of integers as the spine order, an arraylist of \textit{PageEntry} as the edge distribution and a single integer for the number of pages. A so-called \textit{SolutionChecker} is then used to calculate the crossings at the end.

In our opinion it makes more sense to separate the two steps for they can have completely independent implementations and they can be optimized perfectly because of the fine granularity. While the spine-ordering might be enhanced by a local search, the edge partitioning could implement a natural selection algorithm without direct influence on one and each other.

\subsection{Randomized/Multi-Solution Construction Heuristic}
\hspace{0.5cm} As far as randomization goes we implemented it in both steps. Our approach is to generate \textit{k} different solutions, evaluate them and return the best one. In case of the first step we let our \textit{SpineOrderHeuristic} to generate \textit{k} different spine orders where the DFS algorithm is used from above with the modification of picking a random root node. In order to avoid duplicates, only unique \textit{k} spine orders are generated. For each of these spine orders the edge distribution is executed like in the deterministic variant. However edges are not selected from a predefined (sorted \& reduced) list, instead they are picked randomly. Plus we let the edge partitioning loop to run multiple times in succession to further improve the result. Since we've adapted/extended our deterministic approach, the algorithm does not degenerate into random search. After conducting multiple test-runs the randomized version turned out to be performing better in the majority of cases. Despite the overall good results, the huge number of iterations (due to multiple solutions) act as a significant trade-off regarding execution time (and performance). This could be compensated by reducing the number of generated solutions and by applying local search, simulated annealing and/or evolutionary algorithms to improve the few existing ones.

\section{Experimental Setup}
\hspace{0.5cm} The heuristics were tested on a Windows 10 PC with Intel Core i7 870 (4 cores, 2.93GHz) and 8GB RAM, implemented in Java without external libraries. All instances received a CPU time limit of 15 minutes. If the running time exceeds the limit (checked on each iteration), the current best solution is returned. The number of spine order solutions and iterations in the partitioning were allocated dynamically depending on the "weight" of the instance. Instances 1-5 always generated as many different spine order solutions as the number of vertices they had. Instance 6 only randomized 2 spine orders, while the rest (7-10) did 5 different ones. For fairness, we allowed the deterministic algorithm to make multiple iterations through the edges as well.
\newpage
\section{Results}
\subsection{Results of Deterministic Construction Heuristic}
{
\begin{tabular}{l*{6}{c}r}
	Instance & Crossings & Time \\
	\hline
	automatic-1 & 17 & 20 ms \\
	automatic-2 & 10 & 33 ms \\
	automatic-3 & 78 & 78 ms \\
	automatic-4 & 6 & 36 ms \\
	automatic-5 & 20 & 40 ms \\
	automatic-6 & 3'708'167 & 450'273 ms (7.50455 min) \\
	automatic-7 & 8806 & 2523 ms \\
	automatic-8 & 208'775 & 297'682 ms  (4.961 min) \\
	automatic-9 & 31817 & 318'593 ms (5.309 min) \\
	automatic-10 & 17057 & 72813 ms (1.213 min) \\
\end{tabular}
}


\subsection{Results of Randomized Construction Heuristic}
{
\begin{tabular}{l*{6}{c}r}
	Instance & Best Cr. & Avg Cr. & Std. Dev. Cr. & Avg. Time & Std. Dev. Time & Runs \\
	\hline
	automatic-1 & 9 & 11.2 & 2.52 & 209.6 ms & 70.80 ms & 15 \\
	automatic-2 & 3 & 3.8 & 1.13 & 282.8 ms & 12.6 ms & 15 \\
	automatic-3 & 44 & 50.4 & 3.53 & 2678.83 ms & 76.86 ms & 15 \\
	automatic-4 & 2 & 3.9 & 2.38 & 1122.16 ms & 63.26 ms & 15 \\
	automatic-5 & 8 & 11 & 4.16 & 1695.83 ms & 87.69 ms & 15 \\
	automatic-6 & 2'021'124 & 2'859'051 & 371980.26 & 699130.16 ms (11.65 min) & 24889.06 ms & 15 \\
	automatic-7 & 4993 & 6162.5 & 1015.28 & 16667.5 ms & 139.7 ms & 15 \\
	automatic-8 & 198'957 & 203'191.87 & 4353.03 & 785092.5 ms (13.08 min) & 11352.10 ms & 15 \\
	automatic-9 & 31350 & 37752.77 & 3165.16 & 808543.66 ms (13.47 min) & 12397.69 ms & 15 \\
	automatic-10 & 15220 & 15873.27 & 321.92 & 491439.83 ms (8.19 min) & 3254.12 ms & 15 \\
\end{tabular}
}

\end{document}
