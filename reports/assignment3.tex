\documentclass[11pt]{article}
\usepackage{outline}
\usepackage{pmgraph}
\usepackage{listings}
\usepackage[utf8]{inputenc}
\usepackage[normalem]{ulem}
\usepackage{amsmath}
\usepackage{algorithm}
\usepackage[noend]{algpseudocode}
\usepackage{caption}
\usepackage{mathtools}
\def\BState{\State\hskip-\ALG@thistlm}
\makeatother
\renewcommand{\algorithmicrequire}{\textbf{Input:}}
\renewcommand{\algorithmicensure}{\textbf{Output:}}
\DeclarePairedDelimiter\ceil{\lceil}{\rceil}
\DeclarePairedDelimiter\floor{\lfloor}{\rfloor}
\title{Programming Assignment 3\\ \vspace{10px} \large Heuristic Optimization Techniques}
\author{David Molnar, 1326891\\ Daniel Füvesi, 1326576\\Gruppe 9}
\date{\today}
%--------------------Make usable space all of page
\setlength{\oddsidemargin}{0in}
\setlength{\evensidemargin}{0in}
\setlength{\topmargin}{0in}
\setlength{\headsep}{-.25in}
\setlength{\textwidth}{6.5in}
\setlength{\textheight}{8.5in}
%--------------------Indention
\setlength{\parindent}{0.5cm}

\begin{document}
\lstset{language=Java}
%--------------------Title Page
\maketitle
 
%--------------------Begin Outline
\section{Advanced Local Search}
\subsection{General Variable Neighbourhood Search}
\hspace{0.5cm} In the third assignment we chose and developed the General Variable Neighbourhood Search (GVNS) for our probleme instance. GVNS combines the Variable Neighborhood Descent (VND) with the Basic Variable Neighborhood Search (BVNS). To understand how our algorithm works we may discuss the two parts of it. The VND part aims to find the local minimum not only in one neigbourhood but it returns the local optimum with respect of all neigbourhood structures in the set. The other part (BVNS) is "combination of deterministic and stochastic moves in the search
space". It iterates through the neighbourhood structure set, generetes a random solution in the neighbourhood and checks if it was a better solution or not. So, the combination of this to approaches uses two sets of local searches, that should not be identical. Therefore we developed in additon few more local searches. \\

\subsection{Additional local searches}
\hspace{0.5cm} In order to easily have more local searches, we did not have to invent new neigbourhood structures, but only scaling them. So, in addition to the existing neigbourhoods (Single Edge Move, Node Swap, Node Edge Move) we developed the Double Edge Move and the Double Node Swap.\\

\section{Experimental Setup}
\hspace{0.5cm} The GVNS with randomized construction heurictic was tested on a Windows 10 PC with Intel Core i7 870 (4 cores, 2.93GHz) and 8GB RAM, implemented in Java without external libraries. For the testing of the GVNS with deterministic conctruction heuristic we used a Windows 10 PC with Intel Core i7 3630QM (4 cores, 2.4GHz) and 8GB ram. All instances received a CPU time limit of 14 minutes. If the running time exceeds the limit (checked on each iteration), the current best solution is returned. Since we have five different local searches, there is lot of combinations in the two sets of GVNS. We tested many of it to get the proper model to test with. During the evaulation of these tests we found out that the order of the local searches within the sets does have an effect on the quality of the solution. At the end we came up with \{Single Edge Move, Node Edge Move, Double Node Swap\} and \{Single Edge Move, Node Swap, Double Edge Move\} for the two sets in the algorithm. We used this model for all the test cases later. For each construction and randomized construction heuristic we tested the algorithm with all step functions with 15 runs. We used incremental evaluation in all of the local searches. For instance when moving an edge from a page to another, we only calculate the deference for the first page, and for the second we only calculate the new crossings and add it to the existing number of crossings.\\
\section{Results}
\hspace{0.5cm} Due to the high amount of possible combinations of local search sets within the algorithm we invested a lot of time in testing. As already mentioned we used the same model for our test cases. It showed that for the test instances 1, 2, 4, 5 both constuction types with any step functions gave the best solutions. In the second half of the test cases we noticed, that the algorithm did not have enough time to terminate. Since the algorithm starts with a gerenation of a random solution, the quality of this initial solution has a great effect on the solution. Tests showed that we can reach the best solutions with the random step function. For the higher instances (6-10) the first and best step functions resulted relatively constant (bad) results. This causes the quitly big differences between the average and the best solutions.\\
\newline
Abbreviations: (step\_function)
\begin{itemize}
	\item \textbf{b} ... best step function
	\item \textbf{f} ... first step function
	\item \textbf{r} ... random step function
\end{itemize}
\subsection{Results with Deterministic Construction Heuristic}


\begin{tabular}{l*{6}{c}r}
	Instance & Crossings B4 GVNS & Best Crossings & Avg. Crossings & Time & Runs \\
	\hline
	automatic-1 & 18 & 9 (\textbf{b}) & 9 & 1414 ms & 3x15 \\
	automatic-2 & 12 & 0 (\textbf{f}) & 0 & 45 ms & 3x15 \\
	automatic-3 & 127 & 22 (\textbf{b}) & 27,4 & 650'977 ms & 3x15 \\
	automatic-4 & 20 & 0 (\textbf{b}) & 0 & 149 ms & 3x15 \\
	automatic-5 & 23 & 0 (\textbf{b}) & 0,4 & 5724 ms & 3x15 \\
	automatic-6 & 5'751'426 & 1'998'401 (\textbf{r}) & 4'494'828 & 842'376 ms (!) & 3x15 \\
	automatic-7 & 9872 & 7276 (\textbf{r}) & 8345 & 840'076 ms (!) & 3x15 \\
	automatic-8 & 751'523 & 200'994 (\textbf{r}) & 546'144 & 840'498 ms (!) & 3x15  \\
	automatic-9 & 149'327 & 28'679 (\textbf{r}) & 106'653 & 840'445 ms (!) & 3x15 \\
	automatic-10 & 75255 & 16'251 (\textbf{r}) & 48'560 & 840'213 ms (!) & 3x15  \\
\end{tabular}



\subsection{Results with Randomized Construction Heuristic}

\begin{tabular}{l*{6}{c}r}
	Instance & Avg. Crossings B4 GVNS & Best Crossings & Avg. Crossings & Avg. Time & Runs\\
	\hline
	automatic-1 & 20 & 9 (\textbf{b}) & 9 & 36010 ms & 3x14 \\
	automatic-2 & 12 & 0 (\textbf{f})& 0 & 1149 ms & 3x14 \\
	automatic-3 & 151 & 21 (\textbf{r})& 28 & 478173ms & 3x14 \\
	automatic-4 & 18 & 0 (\textbf{f})& 0 & 6760 ms & 3x14 \\
	automatic-5 & 23 & 0 (\textbf{r})& 0.90 & 206990 ms & 3x14 \\
	automatic-6 & 9'813'265 &2'280'656 (\textbf{r})& 7'393'025 & 844817 ms (!) & 3x14 \\
	automatic-7 & 24'935 & 7803 (\textbf{r})& 8505 & 840158 ms (!) & 3x14 \\
	automatic-8 & 1'543'365 & 194'286 (\textbf{r})& 1'102'975 & 841170 ms (!) & 3x14 \\
	automatic-9 & 933'825 & 34'563 (\textbf{r})& 622'879 & 841333 ms (!) & 3x14 \\
	automatic-10 & 186'191 &15'831 (\textbf{r})& 100'160 & 840415 ms (!)  & 3x14 \\
\end{tabular}

\end{document}
